\documentclass[parskip=half,headsepline,footsepline,titlepage]{scrartcl}

\usepackage[ngerman]{babel}
\usepackage{ngerman}
\usepackage[utf8]{inputenc}

\usepackage[
    type={CC},
    modifier={by-sa},
    version={4.0},
]{doclicense}
%\usepackage{fancyhdr}


\usepackage[bf]{caption}

\usepackage[T1]{fontenc}
\usepackage[sfdefault]{FiraSans}
%\usepackage{mathpazo}
%\usepackage{newtxsf}
\usepackage{color}

\usepackage[a4paper,margin=25mm]{geometry}


\usepackage[tocflat]{tocstyle}
%\usepackage{glossaries}

\usepackage{footnote}
\makesavenoteenv{minpage}   % If you want to include minipages.
\makesavenoteenv{itemize}
\makesavenoteenv{enumerate}

\usepackage{booktabs}
\usepackage[hidelinks]{hyperref}
%\usepackage[ngerman, nameinlink]{cleveref}

%\usepackage{chemformula}
%\usepackage{chemfig}
\usepackage{siunitx}
\sisetup{
detect-weight = true,
locale = DE,
}

\usepackage{ulem}
\usepackage{eurosym}
%\usepackage{unicode-math}
\usepackage{graphicx}
\graphicspath{{./images/}}

%\usepackage{wrapfig}
%\usepackage{subfig}

%\usepackage{marginnote}
%\newcommand{\achtung}{\marginnote{\includegraphics[height=1cm,width=1cm]{Achtung.png}}}
%\newcommand{\aetzend}{\marginnote{\includegraphics[height=1cm,width=1cm]{Aetzend.png}}}
%\newcommand{\gesundheit}{\marginnote{\includegraphics[height=1cm,width=1cm]{Gesundheit.png}}}


\title{Maker-/Fab-/Inno-/StartUpLab ``BauRaum'' an der BHT}
\author{Tasso Mulzer}
 

\begin{document} 
%\setcounter{secnumdepth}{3}
%\setcounter{tocdepth}{2}
\date{14. September 2020}
\maketitle
\doclicenseThis


% %\newpage
\pagestyle{empty}
Stand vom \today. Das Dokument steht unter folgender URL zur Verfügung:

\url{https://github.com/tmulzer/BauRaum/}
\tableofcontents
\cleardoubleoddpage

\pagenumbering{arabic}
\pagestyle{plain}

\section{MakerSpace, FabLab, StartUp-Lab, Innovation-Lab}
Ein MakerSpace \footnote{Umgangssprachlich mischen sich teilweise die Bezeichnungen MakerSpace, HackerSpace, FabLab, Innovations-Labor oder StartUp-Labor und anderen, auch wenn die unterschiedlichen Bezeichnungen jeweils leicht unterschiedliche Ausrichtungen kennzeichnen.} bietet den Nutzern einen Raum, in dem sie kreativ und interdisziplinär am Gerät entwickeln und arbeiten können.
Zum einen bietet es die Gelegenheit, die in Vorlesungen erlernte Theorie in vielfältiger Weise an der Praxis zu erleben und zu testen, zum anderen wird interdisziplinäre Kommunikation und Zusammenarbeit gefördert.

Als weiterer Aspekt ist ein MakerSpace auch gut als Anlaufpunkt geeignet, um externe Interessenten in die Hochschule zu bringen und gerade während der Orientierungsphase junge Menschen auf die Entwicklungs-Möglichkeiten und den Praxisbezug in unserer Hochschule aufmerksam zu machen.

Zu guter Letzt werden MakerSpaces oft auch als Innovations- oder StartUp-Lab bezeichnet und bieten besonders jungen Gründern die Möglichkeit, in einer motivierenden Umgebung schnell und unkompliziert zu ersten Prototypen der entstehenden Produkte zu gelangen und dabei einen interdisziplinären Austausch mit anderen Nutzern sowie den Betreuern zu pflegen. Daher bietet sich eine enge Zusammenarbeit mit dem Bereich Technologietransfer / Gründerscout an und das Labor sollte die dort auftretenden Bedürfnisse berücksichtigen und mit abdecken.

\subsection{Unterscheidung MakerSpace / FabLab}

Ein MakerSpace (im englischen “HackerSpace”) ist eine offene Werkstatt, in welcher Menschen mit gemeinsamen Interessen (Computer, Fertigung, Technologie, Wissenschaft, digitale oder elektronische Kunst) sich treffen, vernetzen und zusammenarbeiten.

FabLab bezeichnet kurz zusammengefasst einen MakerSpace-Plus mit bestimmten Schwerpunkten. Im Fablab ist z.B. ein deutlicher Schwerpunkt auf \emph{digitale} Fertigung, Dokumentation und Bildung gelegt, Handwerkzeuge treten etwas in den Hintergrund, sind jedoch auch vorhanden\footnote{Der Anspruch als ingenieursbildende Hochschule sollte vor diesem Hintergrund in jedem Fall die Anerkennung als FabLab sein.}.

Kern eines FabLabs ist die Möglichkeit, dort das von Neil Gershenfeld initiierte Curriculum “How to make (almost) anything” zu absolvieren\footnote{Derzeit besteht großer Bedarf, dieses Curriculum an die Gegebenheiten der deutschen Bildungslandschaft anzupassen. Wenn wir nicht zu lange warten, können wir daran mitwirken und Präsenz zeigen.}. 

Die Ausstattung der FabLabs orientiert sich an einer gemeinsamen Liste von Geräten, die zur Verfügung stehen sollen. Einige der Geräte sind an der BHT bereits vorhanden.
Verschiedene Akteure an Berliner Hochschulen haben sich dazu bekannt, je Hochschule eine gemeinsame Grund-Ausstattung zu schaffen, die übergreifend möglichst kompatibel ist und sich an der gemeinsamen Ausstattung der FabLabs orientiert. Je Hochschule kann zusätzlich eine (oder mehrere) Spezialisierung erfolgen - an der BHT bietet sich hier z. B. die Konzentration auf die Fertigung von PCBs und elektronischen Baugruppen an. Die BHT ist in diesem Bereich besonders gut ausgestattet.


\subsection{Die Fab-Charta}
FabLabs haben sich eine gemeinsame Charta gegeben:
\begin{quote}
\subsection*{Mission}

Fab Labs sind ein globales Netzwerk von lokalen Werkstätten. Sie fördern Erfindungen, indem sie Individuen die Werkzeuge für eine digitale Fertigung zugänglich machen.

\subsection*{Zugang}

Du kannst das Fab Lab nutzen, um fast alles zu machen (außer Dinge, die andere verletzen); du musst lernen, Dinge selbst zu machen, und du musst dir den Gebrauch des Fab Labs mit anderen Nutzern und Nutzungsarten teilen.
\subsubsection*{Bildung}

Das Training im Fab Lab basiert darauf, Projekte durchzuführen und von Mentoren zu lernen; wir erwarten, dass du dich an der Dokumentation und dem Anleiten von anderen beteiligst.

\subsection*{Verantwortung}

Du bist verantwortlich für:
\begin{description}
\item[Sicherheit] – also zu wissen, wie du arbeiten musst, ohne Menschen oder Maschinen Schaden zuzufügen;
\item[Aufräumen] – also das Fab Lab sauberer zu hinterlassen, als du es vorgefunden hast;
\item[Betrieb] – also beim Warten und Reparieren mitzuhelfen und Bescheid zu sagen, wenn es Probleme mit Maschinen gibt, Materialvorräte zur Neige gehen oder Unfälle passieren;
\item[Wissen] - also zur Dokumentation beizutragen und an der Wissensvermittlung mitzuwirken;
\item[friedliche Nutzung] – also darauf zu achten, dass im Lab keine Waffen oder Waffenteile angefertigt und vom Lab aus keine Konstruktionsdateien für Waffen oder Waffenteile verbreitet werden.                                                                                                                                                                                           \end{description}

\subsection*{Geheimhaltung}

Konstruktionen und Verfahren, die im Fab Lab entwickelt wurden, müssen für den persönlichen Gebrauch durch andere zugänglich bleiben. Abgesehen davon können geistige Eigentumsrechte an Konstruktionen und Verfahren geschützt werden.

\subsection*{Geschäft}

Kommerzielle Aktivitäten können in Fab Labs gestartet werden, aber sie dürfen den offenen Zugang für andere nicht behindern; wenn sie zunehmen, sollten sie eher außerhalb des Fab Labs weiter verfolgt werden. Außerdem sollen all die Erfinder, Fab Labs und Netzwerke, die zu ihrem Erfolg beitragen, von ihnen profitieren können.
\end{quote}

\section{Anbindung}
Zunächst soll die Verortung des Labors beschrieben werden. An verschiedenen Hochschulen gibt es unterschiedliche Ansätze, um das Verhältnis zwischen Hochschule und FabLab\footnote{Die genaue Bezeichnung als MakerSpace, FabLab, Innovation \& Research-Lab oder wie auch immer sollte noch gemeinsam definiert werden. Am Ende beschreiben die verschiedenen Bezeichnungen meist im Kern das Gleiche. Um die Lesbarkeit zu gewährleisten, wird in diesem Text ``FabLab'' verwendet - wenn nicht spezifisch eine andere andere Ausrichtung gemeint ist.} zu gestalten.

\subsection{Eingebettet in die Beuth-Hochschule}
Teilweise\footnote{z.B. Uni Weimar} entstanden bisher HackSpaces mehr oder weniger im Alleingang aus der Studierendenschaft heraus. 

Zum einen resultiert daraus Konfliktpotential zwischen Hochschule und dem Kreativraum, weil wichtige Strukturen zur Steuerung fehlen, zum anderen fehlt in diesen Fällen leider auch häufig eine erkennbare Professionalität.

In anderen Fällen wurde durch die Hochschul-Leitung oder andere Akteure der ``oberen Ebenen'' ein solcher Raum eingerichtet - jedoch, ohne die angedachten Nutzer von Anfang an einzubeziehen. Diese Räume werden oft entweder nicht ``fertig`` oder oft nur zögerlich angenommen.

Beides soll in unserem Ansatz nicht der Fall sein. Das FabLab soll aus der Beuth Hochschule für Technik heraus entstehen und von Anfang an durch sämtliche in Frage kommenden Ebenen von der Studierendenschaft bis hin zum Präsidium gemeinsam getragen, genutzt und gestaltet werden.

Gleichzeitig soll das Labor zu einem hochschulintern sinnvoll nutzbaren Projektarbeitsraum werden, der auf die Bedürfnisse der Hochschule eingeht und sich beständig fortentwickelt.

Als gutes Vorbild kann das VINN:Lab an der TH Wildau betrachtet werden. In den letzten Jahren bestand ein guter Kontakt zu weiteren FabLabs innerhalb von Hochschulen (z.B. Wildau, TU Berlin, Siegen, Aachen, Dresden, ...), so dass dort gesammelte und bei verschiedenen Treffen\footnote{Besonders die seit 2017 jährlich stattfindende Fab:Universe Konferenz war regelmäßig eine hervorragende Gelegenheit zur Vernetzung mit anderen FabLabs innerhalb der Hochschulen.} gebündelte Erfahrungen im Aufbau berücksichtigt werden können.


\subsection{Zusammenarbeit innerhalb der BHT}

Das FabLab wird in der aktuellen Planung nicht innerhalb eines Fachbereiches angesiedelt sein, da nur so sichergestellt ist dass alle 8 Fachbereiche der BHT und das FSI gleichberechtigt auf das Labor zugreifen können. An anderen Hochschulen hat sich eine Einrichtung als zentrale Stelle - z.B. angegliedert an die Bibliothek bewährt\footnote{Als Beispiel einer gelungenen Anbindung als zentrale Einrichtung kann der SLUB-MakerSpace an der sächsischen Landesbibliothek – Staats- und Universitätsbibliothek dienen.}. An der Beuth Hochschule ist die Abteilung Technologie-Transfer (TT) als zentrale Schnittstelle strukturell besonders gut geeignet. Eine Zusammenarbeit mit anderen Einrichtungen der Beuth Hochschule bietet sich an und sollte von vorneherein eingeplant werden.
In vielen positiven Beispielen kommt FabLabs innerhalb von Hochschulen auch eine zentrale Vermittlungs-Rolle zu - durch die fachbereichsübergreifende Aufstellung ist das FabLab oft auch gut als Koordinationsstelle\footnote{An der TU Berlin wurde für diese Rolle der Begriff ''Spinne im Netz`` geprägt.} für fachbereichsübergreifende Aktivitäten und Fragestellungen einsetzbar und fördert in dieser Eigenschaft die interdisziplinäre Zusammenarbeit zwischen Studiengängen und Fachbereichen.


\subsubsection{BeuthBox / Medien- und Didaktikzentrum / InnoVision-Lab}
Bereits 2017 wurde auf der ersten Veranstaltung ''FabUniverse`` an der TH Wildau festgestellt, dass in FabLabs ein großer Bedarf an Dokumentation und Präsentation der Verfahren und Ergebnisse auf dem Weg zu Prototypen besteht. Daher bietet sich eine enge Zusammenarbeit mit der BeuthBox als Zentrum für Digitale Medien (VR \& AR) an.
Aufgrund von Erfahrungen aus dem Zentrum ''ProLehre - Medien und Didaktik``an der TU München verdichtete sich Ende 2019 die Idee, das FabLab sehr eng an das Medienzentrum der Beuth Hochschule (in Form der BeuthBox) anzugliedern und die beiden Einheiten zu einer gemeinsamen Einrichtung zu verschmelzen. Hiervon werden verschiedene positive Effekte erwartet:
\begin{itemize}
 \item Didaktische Inhalte (auch aus anderen Bereichen der Lehre) können im FabLab oft praxisnah aufbereitet medial erfasst werden.
 \item Der Erstellungsprozess von Prototypen kann ''ad hoc'' aufgezeichnet und im Nachlauf aufbereitet werden - dadurch verringert sich der Dokumentations-Aufwand deutlich.
 \item Oft zur Produkt-Präsentation benötigte Video- oder 3D-Darstellungen können unkompliziert und in hervorragender Qualität innerhalb des Bereiches erstellt werden - die Medienspezialisten kennen das Projekt durch die räumliche Nähe oft auch besser als ein erst am Ende des Projektes hinzugezogenes Filmteam.
 \item Durch die mediale Begleitung im Labor erlangen die Projektbeteiligten Erfahrung im Umgang mit Aufzeichnungs- und Präsentations-Situationen -- hierbei gewinnen sie bereits früh eine später bei Produktpräsentationen (/Pitches) sehr hilfreiche Erfahrung und Sicherheit im Auftritt.
\end{itemize}


\subsubsection{HackSpace (FB VI)}
Neben dem FabLab, in dem die Projekte üblicherweise einen Hardware-Schwerpunkt haben, bietet sich der parallele Aufbau eines HackSpace gemeinsam mit dem FB VI an. Üblicherweise haben (auch wenn es Schnittmengen gibt) Software-Projekte andere Bedürfnisse als Hardware-Projekte. Das Gesamt-Konzept sollte in jedem Fall gemeinsam mit dem FB VI entwickelt werden und entweder bewusst beide Bereiche interdisziplinär in einem Raum verbinden - oder in einer engen Zusammenarbeit, aber doch getrennt gestaltet werden.

\subsubsection{Holzwerkstatt (FB VIII / FB IV)}
Die vorhandenen Anlagen im FB VII ermöglichen derzeit eine gute Bearbeitung von Metall und / oder Kunststoffen. Holzbearbeitung ist im Laborbereich aktuell kaum sinnvoll möglich und wird auch nur schwer integrierbar sein. Daher sollte in der Nähe des FabLabs eine Holzwerkstatt mit ähnlichem Konzept entstehen, um die Bedürfnisse der kreativen Arbeit vollständig abzudecken.

%\subsubsection{Digitalwerkstatt (FB IV)}


\subsection{Hilfskonstrukt ''eingetragener Verein``}
Verschiedene Dinge die beim Betrieb eines FabLabs notwendig werden, sind im Rahmen der Hochschule nur schwer bis gar nicht abbildbar. Um an dieser Stelle flexibel und gleichzeitig sinnvoll, rechtssicher und korrekt handeln und den Betrieb des FabLab transparent gestalten zu können, ist die Anbindung eines weiteren eingetragenen Vereins (z.B.: ``Verein der Freunde und Helfer des BauRaums'') hilfreich. Dabei muss auf ein sinnvolles Miteinander (Zusammenarbeit / Abgrenzung) mit dem Förderverein der Beuth Hochschule geachtet werden.
Der Verein soll und darf nicht außerhalb der Hochschule stehen - der Verein soll vielmehr dauerhaft innerhalb der Hochschule aktiv sein und bleiben.

Dem an das FabLab und die Hochschule angebundenen Verein kommen verschiedene Aufgaben zu, die gemeinsam noch genauer zu definieren sind. Folgende Aufgaben seien exemplarisch genannt und erläutert.

\subsubsection{Anbindung von Sponsoren / Unterstützern\label{sssec:sponsoren}}
Allgemein können Sponsoren und Unterstützer auch über den Förderverein der Beuth Hochschule für Technik angebunden werden - einige (vermutlich die meisten der im Kern wichtigen) Unterstützer werden jedoch sicher lieber spezifisch die Aktivität ``FabLab'' fördern als eher unspezifisch die Beuth Hochschule insgesamt. Eine wichtige Aufgabe des Vereins wird es daher sein, diese Unterstützer gezielt zu sammeln, am Ball zu halten und in die Entwicklung des Labors einzubinden.

Ein weiterer Punkt sind verschiedene Förderprogramme, die speziell auf gemeinnützige Einrichtungen zugeschnitten sind - um an diesen Förderprogrammen teilnehmen zu können gilt oft eine als gemeinnützig anerkannte Vereinsstruktur als Voraussetzung.

Der Verein kann gleichzeitig als ``Heimat'' für ehrenamtliche Helfer und Unterstützer sowie die später beschriebenen ``Maschinen-Engel'' (s. Kapitel \ref{sssec:maschinenengel} auf Seite \pageref{sssec:maschinenengel}) dienen, die innerhalb des Vereins strukturiert und klar an das Labor angebunden werden können.


\subsubsection{Versicherung von und gegenüber Externen im Labor}
Das Labor soll in bestimmten (noch festzulegenden) Zeiten\footnote{Angedacht ist zunächst etwa ein Tag pro Woche - ähnlich wie es die TH Wildau praktiziert.} auch für externe Nutzer\footnote{Dabei sei primär an Schüler von Oberschulen und Gymnasien gedacht, um diese frühzeitig für die Beuth Hochschule für Technik zu gewinnen. Siehe Kapitel \ref{p:zielgruppe-extern} auf Seite \pageref{p:zielgruppe-extern}} geöffnet sein (s. Kapitel \ref{p:zielgruppe-extern} auf Seite \pageref{p:zielgruppe-extern}). 

Da die externen Nutzer nicht als Studierende über die Beuth Hochschule für Technik versichert sind, ist der Verein (z.B. über eine Mitgliedschaft des Vereins im Verband der offenen Werkstätten~--~VoW) eine Möglichkeit, diese Lücke zu schließen.

Dabei ist zum einen eine Unfallversicherung der Nutzer, zum anderen eine Haftpflichtversicherung gegenüber den Nutzern zu betrachten.

Selbstverständlich sollte auch darauf geachtet werden, dass die Nutzer des Labors -- falls ein entsprechendes Risiko entsteht -- selbst mit einer privaten Haftpflichtversicherung ausgestattet sein sollen / müssen\footnote{Eine private Haftpflichtversicherung von Nutzern des FabLab ist sicher nicht in jedem Fall zwingend notwendig -- gerade bei einer eigenständigen Bedienung von teureren Maschinen und Ausrüstungsgegenständen sollte das jedoch zu den Voraussetzungen zur Nutzung gehören.}. Letzteres lässt sich z.B. sinnvollerweise in die Nutzungsbedingungen für risikobehaftete / hochwertige Ausrüstung\footnote{Als Richtlinie für die Voraussetzung einer Haftpflichtversicherung kann z.B. ``alles ab einem Wiederbeschaffungswert über 150 \euro'' gelten.} wie z.B. große Fräsen oder teure und empfindliche Messgeräte aufnehmen.

\subsubsection{Grundlage: Registrierte Vereinigung - analog zur TU Berlin}
Um die Einbettung des FabLab incl. des unterstützenden Vereines in die Beuth Hochschule zu strukturieren und schriftlich zu fixieren, kann auf ein Konstrukt wie die an der TU Berlin bewährte ``Registrierte Vereinigung''\footnote{Als Beispiel einer gut funktionierenden registrierten Vereinigung an der TU Berlin kann der Amateurfunk-Club DK0TU gelten, der auch Angehörige der Beuth Hochschule dort einbindet.} zurückgegriffen werden.
Im Rahmen einer solchen Registrierung müssen sämtliche Aspekte des Verhältnisses zwischen der Beuth Hochschule für Technik und dem FabLab definiert und festgehalten werden\footnote{Unter anderem sollte in diesem Zusammenhang z.B. eine kostenfreie Nutzung der Laborflächen geregelt werden. Anfangs aufzubringende Mietverpflichtungen würden den Aufbau vermutlich im Keim ersticken.}.
%ToDo: Fußnote drin lassen?? - passt sicher nicht in allen Zusammenhängen.


\section{Struktur}

\subsection{Kerngebiete}
Um eine klare Struktur zu gestalten und die Ausrichtung des Labors erkennbar zu machen, ist eine Konzentration auf wenige Kerngebiete sinnvoll. Das Labor kann -- und soll -- keine ``eierlegende WollMilchSau'' werden sondern einer klaren und erreichbaren Zielsetzung folgen.

\subsubsection{Montage / Fehlersuche}
Kerngebiet des Labors ist der Aufbau, der Zusammenbau und die parallele Fehlersuche von mechatronischen Systemen. Idealerweise wurden die Konzepte bereits in anderen Laboren (KoCAD / GOS / Digitaltechnik / ...) entwickelt und werden im FVM-Labor dann zusammengesetzt und montiert. 

Real wird zum einen kaum ein Konzept das ins Labor kommt wirklich ``fertig'' sein -- zum anderen wird im Aufbauprozess üblicherweise so schnell zwischen den verschiedenen Tätigkeiten gesprungen, dass sicher teilweise auch Entwicklungsprozesse im FVM-Labor stattfinden werden. Die Ausstattung des Labors soll sich jedoch auf den Aufbau und die Fehlersuche konzentrieren.

In diesem Kerngebiet soll sowohl eine Entwicklung von strukturierten Montageprozessen -- gleichzeitig aber auch eine kreative Arbeitsweise -- gefördert werden.

Es soll durchaus erlaubt sein, auch unkonventionelle und innovative Herangehensweisen im Rahmen des technisch möglichen\footnote{Voraussetzung ist natürlich, dass die neuen Herangehensweisen gefahrlos und kontrollierbar umsetzbar sind.} zu testen.

\subsubsection{Umgang mit Ideen}
Im Kern ist es wichtig, Ideen zum einen zu sammeln, herauszuarbeiten und zu entwickeln. Zum anderen geht es zentral auch darum, Ideen nicht nur zu sammeln, sondern auch gezielt herauszugreifen und (gemeinsam) zur Umsetzung zu bringen.

\paragraph{Aufnehmen von Ideen}
Derzeit existiert eine kleine Jira-Datenbank (``Pool of good Ideas``), in der Ideen gesammelt werden können. An dieser Stelle sollte zeitnah ein dedizierter ''Harbour of good Ideas`` -- d.h. ein Verfahren geschaffen werden, die folgende Punkte gewährleistet:

\begin{itemize}
 \item Ideen sollen zunächst vertraulich erfasst und eingestellt werden können. An dieser Stelle muss eine Schnittstelle existieren, die zunächst ermöglicht, eine Idee konkret in Text-/Bild-/Dateiform anzulegen und zu speichern. Zugriff auf die einzelne Idee darf in diesem Moment ausschließlich der Ideen-Geber selbst haben.
 \item Es muss zu jedem Zeitpunkt klar nachvollziehbar und dokumentiert sein, wer die Idee hatte - d.h. wem das geistige Eigentum an dieser Idee zuzuschreiben ist. \footnote{Der Labor-Administration kommt die Aufgabe zu, bei den Ideen-Gebern Bewusstsein für die Möglichkeiten sowohl des Ideen-Schutzes als auch der Ideen-Lizenzierung und -Verwertung zu schaffen. Zur Beantwortung von Detailfragen zu diesen Themen müssen bei Bedarf Fachleute aus dem Netzwerk des Labors vermittelt werden.}
 \item Der Ideen-Geber muss die Möglichkeit haben, die erfasste Idee gezielt und unter für jeden Beteiligten jederzeit klar erkennbaren und durchsetzbaren Bedingungen (Lizenzierung / Weitergabevoraussetzungen) zu teilen und Einzelpersonen, Gruppen oder der Allgemeinheit zur Verfügung zu stellen.
 
\end{itemize}

\paragraph{Umsetzen von Ideen}
Es genügt auf Dauer nicht, Ideen nur um ihrer selbst zu sammeln -- Ideen verhalten sich an der Stelle nicht wie Briefmarken. Eine Idee entfaltet sich erst, wenn sie konkret erprobt, entwickelt und am Ende auch umgesetzt wird.
Sinn und Zweck des Labors ist es dabei auch, die Personen mit eigenen Ideen mit weiteren Personen zu vernetzen, die die zur Umsetzung notwendigen Fähigkeiten mitbringen.

\begin{itemize}
\item Nutzer des Labors sollen die Möglichkeit haben, im ''Harbour of good Ideas`` nach Schlagworten zu suchen bzw. Abstracts zu lesen und damit gezielt passende Ideen zu finden -- natürlich nach Freigabe durch den ''Owner``.
\item Die Labormitarbeiter sollen motivieren, gezielt Teams zu formen und einzelne Ideen aus der großen Masse herauszufischen die dann konzentriert angegangen werden. Für die Art und Details der Umsetzung sind die Teams jeweils selbst verantwortlich.
\item Sobald qualifizierte Ergebnisse erreicht werden, sollten die Teams in der Findung von evtl. notwendigen Anschluss-Förderungen durch das Labor-Netzwerk unterstützt werden -- jedoch nur so weit das gewünscht ist und Sinn macht\footnote{Die Akteure im Labor sollten darauf achten, kein ''Helfersyndrom`` zu entwickeln .. die Oma, die nicht über die Straße möchte, wird nicht hinüber getragen.}.
\item Die Labormitarbeiter sollen sich mit dem notwendigen Wissen und den notwendigen Kontakten ausstatten, um gezielt Ansprechpartner für die relevanten Fragestellungen der Nutzer sowohl innerhalb als auch außerhalb der Hochschule zu kennen, bei Bedarf zu finden und zu vermitteln.
\item Vorstellbare Ergebnisse sollen innerhalb und gerne auch außerhalb des Labors in geeigneter Form präsentiert und bekannt gemacht werden. 
\end{itemize}


\subsubsection{Leiterplatten / elektronische Baugruppen}
Leiterplatten sind die Grundlage vieler elektromechanischer Baugruppen und Systeme. Die Leiterplattenfertigung hat sich im Laborbereich an der Beuth Hochschule für Technik seit Jahrzehnten etabliert, entspricht nach wie vor dem Stand der Technik und sollte in das Konzept eingebunden sein.

Die Leiterplatten- und Baugruppenfertigung beinhaltet eine in der Lehre besonders wertvolle Vielfalt der verschiedensten Fertigungsverfahren in einem vergleichsweise statischen Fertigungsprozess.

\subsubsection{Mechanische Fertigung}
Wie bereits geschrieben, ist die mechanische Bearbeitung von Metallen und Kunststoffen im Labor bereits möglich. Diese Möglichkeiten müssen konsolidiert - und analog zur üblichen Ausstattung eines FabLab ergänzt werden.
Um sinnvoll in mehreren Gruppen arbeiten zu können, muss die Werkzeugausstattung entsprechend ergänzt werden.

\subsection{Ausrüstung}
Die Ausrüstung des Labors soll sich an den Notwendigkeiten zum kreativen und innovativen -- und dabei vor allen Dingen schnellen und professionellen -- Aufbau elektromechanischer Baugruppen ausrichten. Als Orientierung kann die vom MIT veröffentlichte Muster-Inventarliste\footnote{FabLab - Inventory: \url{http://fab.cba.mit.edu/about/fab/inv.html}} für FabLabs dienen. % ToDo: Link als footnote

Bei der Beschaffung der Ausrüstung muss auf eine brauchbare Qualität geachtet werden. Es geht nicht darum, möglichst viel und billiges Werkzeug zu beschaffen, sondern es muss bewusst darauf geachtet werden, eine möglichst sinnvolle und ausgewogene Auswahl von (oft durchaus interessanten) ``billigen'' und professionellen bis hin zu industriell einsetzbaren Werkzeugen herzustellen.

Die Studierenden (und andere Nutzer) sollen im FabLab vor allen Dingen auch den richtigen Einsatz von Werkzeugen unterschiedlicher Qualitätsstufen bewusst lernen und erkennen, um mit dieser Erfahrung sinnvoll -- und letztendlich wirtschaftlich -- agieren zu können.

Die Digitalisierung in der Fertigung soll intensiv genutzt werden. Durch die Digitalisierung sind etliche Fertigungsverfahren auch für Nutzer mit wenig klassischer Maschinenerfahrung verfügbar geworden. Die dadurch entstandenen Möglichkeiten sollen den Nutzern weitestmöglich zugänglich gemacht werden.


\subsection{Öffnungszeiten}

Das Labor soll an 4 Tagen pro Woche wie jedes andere Labor für Lehrveranstaltungen zur Verfügung stehen und genutzt werden. Dabei muss berücksichtigt werden, dass praktische Tätigkeiten oft nur schwer in den getakteten Betrieb nach Stundenplan einzurahmen sind. Oft sind praktische Aufgaben am Ende der geplanten Zeit nicht fertig / fehlerhaft und daher nur schwer sinnvoll zu unterbrechen -- oder sie sind wider Erwarten schon lange vor dem geplanten Ende erledigt. Daher muss in diesem Bereich bei der Zeitplanung von Lehrveranstaltungen eine gewisse Flexibilität erhalten bleiben.

So lange keine geplanten Lehrveranstaltungen im Labor stattfinden, sollen die Räume und Einrichtungen den Studierenden fachbereichsübergreifend zur Verfügung stehen. Die Nutzung muss dabei jedoch an gewisse Regeln (Labor-Ordnung) gebunden sein und kann nicht vollständig frei stattfinden\footnote{Studierende ohne Maschinen-Ausbildung können z.B. die zur Verfügung stehenden Werkzeugmaschinen nicht bedienen. Dazu später mehr.}.

An einem Tag der Woche sollte eine gemischte Nutzung durch externe Gäste und Studierende möglich sein und gefördert werden.

Die Gäste können dabei zum einen freie Interessenten, zum anderen aber auch betreute Gruppen\footnote{Eine Betreuung von externen Gästen kommt zum einen durch BHT-Angehörige, zum anderen aber auch durch weitere externe Akteure (z.B. ChaosMachtSchule e.V. und andere) in Frage.} sein.

Die Öffnungszeiten müssen sich dabei zum einen nach den Lehrveranstaltungen, zum anderen nach der Verfügbarkeit von Labormitarbeitern und / oder verantwortlichen Betreuern\footnote{Ein offener Betrieb ohne Anwesenheit von Professoren, Labormitarbeitern oder anderen, qualifizierten Betreuern kommt nicht in Frage.} richten.

Eine Öffnung an Samstagen (wie z.B. an der TH Wildau) kommt nach aktueller Planung mittelfristig nicht in Frage. Wenn der Betrieb innerhalb der Woche stabil und bewährt läuft, ein dringender Bedarf erkennbar ist und die Rahmenbedingungen es ermöglichen, kann über diese Fragestellung evtl. neu nachgedacht werden.

\subsection{Reservierung von Maschinen und Ausrüstung}
Eine Reservierung von Ausrüstung / Maschinen ist nur in sehr geringem Maße vorgesehen, da jede Reservierungsmöglichkeit der Effizienz bei der Auslastung der Maschinen sowie oft auch der Qualität der hergestellten Dinge entgegenwirkt. 
Zum einen sind die Daten / Modelle, die auf den Maschinen hergestellt werden oft zum Termin noch nicht (oder nicht vollständig) fertig -- in diesem Fall kann der Termin nicht genutzt werden und die Maschine steht still oder es werden ``halbgare`` Dinge hergestellt, um den Termin doch noch zu nutzen.
Zum anderen sind Daten / Modelle oft auch schon (manchmal lange!) vor dem reservierten Termin fertig -- und die Nutzer warten dann unnötig lange auf die (eigentlich freie) Maschinenkapazität.

In der Praxis hat sich ein (moderiertes) First-Come-First-Serve Management der Maschinen bewährt. Eine stillstehende Maschine kann genutzt werden, eine aktuell arbeitende Maschine erst wieder nachdem sie fertig, durch den Nutzer gereinigt, von Labormitarbeitern abgenommen und wieder freigegeben ist\footnote{Ein Hard-/Softwaresystem um diese Nutzung zu strukturieren ist derzeit gemeinsam mit der Uni Siegen in Arbeit.}. Ein Belegen oder Reservieren der Maschinen ist dabei zunächst nicht vorgesehen -- für die Durchführung von geplanten Lehrveranstaltungen kann das vereinzelt jedoch notwendig sein, an dieser Stelle ist dann die Moderation durch Labormitarbeiter und Lehrkräfte erforderlich.

Um die Effizienz der Maschinennutzung in dieser Betriebsart zu fördern, sollte eine Übersicht über die aktuelle und soweit möglich auch für eine kurzfristig\footnote{Eine kurzfristige, maximal ein bis zwei Stunden im Voraus eintragbare ``Reservierung'' kann sinnvoll sein -- etwa, um auf dem Weg zum Labor sicher zu gehen dass die Maschine bei Ankunft dann auch verwendet werden kann.} vorausschauende Maschinennutzung -- durch registrierte Labornutzer -- neben den Öffnungszeiten des Labors im Internet verfügbar und einsehbar sein.

Ein ``Ausleihen'' von Laborausrüstung ist nicht vorgesehen\footnote{Keine Regel ohne Ausnahme - Es geht hier um Werkzeuge und Maschinen. Dinge (z.B. EvalBoards), die im Rahmen der Projektstarthilfe zur Verfügung gestellt werden, sollen natürlich mit nach Hause genommen werden können, um auch dort an den Projekten weiterarbeiten und -tüfteln zu können. Auch ein ``altes'' Oszilloskop steht zur Verfügung, das bereits jetzt zuverlässigen Nutzern ohne privates Gerät zur Ausleihe zur Verfügung steht.} und sollte sehr restriktiv geregelt sein. Es geht schließlich darum, die Nutzer im Labor zu haben und nicht darum, Werkzeuge und Ausrüstung für die private Nutzung zur Verfügung zu stellen.

%\subsubsection{Abhängig von Lehrveranstaltungen und Mitarbeitern}


\section{Personen}

\subsection{Zielgruppe(n)}

Primär soll das FabLab natürlich unseren Studierenden zur Verfügung stehen. Zum Teil in betreuten Lehrveranstaltungen, zu einem großen Teil jedoch auch Studierenden, die im Rahmen des freien Übens ihren Horizont über den Lehrstoff hinaus (oder auch gerade im Bezug auf den Lehrstoff) praktisch erweitern möchten. Die Erfahrung zeigt, dass Studierende mit einer ganz anderen Motivation lernen, wenn sie eigene Ideen verfolgen und anfangen, den gelernten Stoff zu transferieren.
%ToDo: Wikipedia - Transferwissen.

\label{p:zielgruppe-extern}Als weitere Zielgruppe sollen externe Jugendliche (etwa ab der 9. / 10. Klasse) die Gelegenheit bekommen, das Labor zu nutzen. Dies sind junge Menschen, die sich demnächst evtl. für ein Studium entscheiden und bei der Labornutzung eine gute Gelegenheit haben, ``Beuth-Luft'' zu schnuppern. Wenn das Konzept aufgeht und sich diese Jugendlichen bei uns gut aufgehoben fühlen, werden sie sich später auch gerne für ein Studium an der Beuth Hochschule entscheiden.

Die nächste Zielgruppe sind durch den Gründerscout betreute Startups. Die für die erste Zielgruppe wichtige und damit ohnehin vorhandene Ausstattung schafft ein Umfeld. das gut geeignet ist um Gründern die Gelegenheit zu geben, viele ihrer Prototypen im Labor selbst zu erstellen.

Wichtig ist, dass auch Gäste ohne eigene Projekte sich willkommen und wohl fühlen können. Es geht schließlich auch darum, gerade diese Gäste zur Mitarbeit zu animieren und mitzunehmen. Das Wissen und Können dieser Gäste kann den Gästen mit eigenen Projekten oft gerade fehlen.

\subsection{Akteure}
Die im Labor anfallenden Aufgaben werden sich auf verschiedene Akteure verteilen. 

Eine enge Abstimmung und Zusammenarbeit aller Beteiligten ist für den Erfolg des Konzepts unverzichtbar.

\subsubsection{Professoren}

Generell ist die Haupt-Aufgabe der Professoren, das Labor sinnvoll zu nutzen, die Möglichkeiten die das Labor bietet lehrveranstaltungsübergreifend in die Lehre einzubeziehen und die Studierenden zum einen in einzelnen Lehrveranstaltungen im Rahmen ihres Lehrauftrags zu betreuen - aber auch zur sinnvollen freien Nutzung des Labors anzuregen.

Daneben werden verschiedene Professoren selbstverständlich federführend die Leitung des Labors gestalten und die angestrebte Entwicklung steuern, fürdern und nach außen vertreten.

Ob an der Beuth Hochschule für Technik ein ähnliches Konstrukt möglich und sinnvoll ist wie an der TH Wildau, muss evaluiert und entschieden werden. An der TH Wildau wird im Rahmen des dort etablierten FabLabs\footnote{VINN:Lab} in einer eigenen Professur (Prof. Dr. Dana Mietzner) aktiv und eng an das FabLab angebunden über Innovations- und Regionalmanagement geforscht. Das FabLab ist dabei sowohl Forschungsausstattung und Hilfsmittel als auch Forschungsobjekt.

\subsubsection{Lehrbeauftragte}
Genau wie in anderen Bereichen der Hochschul-Lehre sollen zur Ergänzung je nach Bedarf auch Lehrbeauftragte das Labor nutzen. Lehrbeauftrage können zum einen einen engen Bezug zur beruflichen Praxis als auch aktuelles Detailwissen für ihre Themengebiete beisteuern.

\subsubsection{Labormitarbeiter}
Die Aufgaben der Labormitarbeiter werden primär die Unterstützung der Lehrenden im Bereich der praktischen Betreuung der Studierenden liegen. Dazu gehören Beispielsweise die Einweisungen in die Laborarbeit und in einzelne Maschinen und Geräte.
Auch an der Vernetzung des FabLabs mit externen Akteuren und an der Außendarstellung sind die Labormitarbeiter in ihrer Rolle aktiv beteiligt und ergänzen die Reichweite der Professoren.
Neben dem selbstverständlich zu erledigenden Tagesgeschäft (Rechnungen / Bestellungen / ...) ist eine weitere wichtige Aufgabe die praktische Planung, Betreuung und Instandhaltung der Labortechnik sowie die Weiterentwicklung des Laborkonzepts gemeinsam mit den Lehrenden und im Sinne der Studierenden / Nutzer. Dabei kommt den Labormitarbeitern auch die Rolle zu, die praktischen Bedürfnisse der Studierenden aufzunehmen, zu strukturieren und über den Lehr- und Forschungsbetrieb hinausgehend in die Fortentwicklung des Konzepts einzubringen.

Um die zu erwartende Arbeitslast sinnvoll zu bewältigen, sind mindestens zwei (besser drei) aktive Labormitarbeiter notwendig, die sich jeweils fachlich ergänzen und so die gesamte, sowohl im Hochschul-Labor als auch in einem FabLab auftretende Bandbreite an notwendiger Unterstützung für die Nutzer vor Ort gewährleisten.

Um einen Verlust von Fachwissen und Erfahrung zu verhindern, sollen Mitarbeiterstellen frühzeitig wiederbesetzt werden - idealerweise soll eine etwa halbjährige Einarbeitung eines neuen Mitarbeiters durch seinen Vorgänger eingeplant werden.

\subsubsection{Studentische Hilfskräfte}
Studentische Hilfskräfte können Professoren und Labormitarbeiter an verschiedenen Stellen entlasten. Gut eingefahrene und dokumentierte Prozesse können ohne weiteres durch studentische Hilfskräfte weitergegeben und multipliziert werden. Lehrkräften und Labormitarbeitern kann dadurch der Rücken frei gehalten werden, so dass sich diese konzentriert um komplexere und weniger repetitive Aufgaben kümmern können.

Studentische Hilfskräfte können und sollen die Arbeit der Labormitarbeiter nicht ersetzen - sondern je nach Verfügbarkeit und sinnvoller Einsetzbarkeit einfache Aufgaben abnehmen und entlasten. Nicht zu vernachlässigen ist dabei der Lerneffekt (sowohl fachlich als auch persönlich) für die studentischen Hilfskräfte.


\subsubsection{Ehrenamtliche}
Eine Mitwirkung von ehrenamtlichen Akteuren im FabLab kommt im ersten Schritt nur begrenzt in Frage. Sobald sich wichtige Prozesse, Veranstaltungen und Entwicklungen eingespielt haben, können und sollen Ehrenamtliche jedoch gut unterstützen und einen weiteren Mehrwert im Laborbereich liefern.
Oft bringen ehrenamtliche Helfer in einem FabLab eine wertvolle Praxisnähe in das Labor, die sonst alleine innerhalb der Hochschule auf andere Weise nur schwer realisiert werden kann.

Ehrenamtliche Helfer und Unterstützer fördern oft auch eine in besonderer Weise motivierende Atmosphäre im Labor / FabLab.


\label{sssec:maschinenengel}\subsubsection[``Maschinen-Engel'']{``Maschinen-Engel`` \protect\footnote{In Anlehnung an das Konzept der freiwilligen Helfer auf den Congress-Veranstaltungen des Chaos Computer Clubs.} }
Etliche Nutzer werden im üblichen Laborbetrieb manche Maschinen\footnote{Besonders die spanenden Werkzeugmaschinen bedingen in oft eine Ausbildung und Erfahrung, um sinnvoll eingesetzt werden zu können.} nicht sicher und sinnvoll einsetzen können -- manche Maschinen sind einfach nicht dazu geeignet, ohne Ausbildung und / oder Erfahrung betrieben zu werden.
Gleichwohl sind diese Maschinen für die Herstellung von Prototypen oft unverzichtbar und viele Nutzer werden einen Bedarf an Teilen aus diesen Maschinen haben.

Einzelne Unterstützer des Labors mit entsprechender Ausbildung können Lücken an diesen Stellen als ''Maschinen-Engel`` wirkungsvoll schließen.

Sämtliche Arbeiten durch die Labormitarbeiter durchführen zu lassen, bringt diese erfahrungsgemäß schnell an die Grenzen des Machbaren und sorgt für untragbar lange Wartezeiten - sowohl auf einfache Hilfestellung durch die Laboringenieure - als auch auf z.B. Frästeile. An dieser Stelle können zuverlässige (ehrenamtliche) Helfer eingesetzt werden, die die Maschinen gut beherrschen und oft auch gerne an diesen Maschinen arbeiten. Um diese Helfer strukturiert zu organisieren und deren Tätigkeiten zu versichern, ist der oben angesprochene Verein wichtig.

In Frage kommen zunächst besonders intrinsisch motivierte Helfer, die sich dem FabLab verbunden fühlen und ihre Erfahrung und Leistung im Rahmen einer ehrenamtlichen Tätigkeit gerne zur Verfügung stellen. In manchen Fällen kommt evtl. ein Ausgleich z.B. in der Art ''Arbeitsleistung $\Leftrightarrow$ Maschinenstunden`` in Frage.
Erfahrungen aus anderen FabLabs zeigen, dass diese Art der Zusammenarbeit oft gut funktioniert.

Die ''Kunden`` der Maschinen-Engel lernen dabei zwar nur begrenzt den praktischen Umgang mit den Maschinen. Sie üben jedoch, ihre Ideen und Wünsche und Aufgabenstellungen klar, vollständig und verständlich zu kommunizieren. Auf eine fertigungsgerechte Konstruktion und Dokumentation muss in diesen Fällen durch die betreuenden Professoren von vorneherein besonderer Wert gelegt werden.

Es muss auch gewährleistet sein, dass z.B. die Reinigung einer Maschine in diesen Fällen nicht an den ''Maschinen-Engeln`` hängen bleibt. Diese Tätigkeiten sollten -- soweit sinnvoll möglich -- durch die Nutzer durchgeführt werden, die die Leistung in Anspruch nehmen.


\section{Finanzierung}

\subsection{Grundlagenfinanzierung durch die Hochschule}
Auf der Fab:Universe Konferenz in 2019 wurde deutlich, dass die Grundlagenfinanzierung eines Hochschul-FabLabs aus der Hochschule kommen muss. Der Bereich ist eine zentrale Einrichtung der Hochschule und daher selbstverständlich auch so zu finanzieren. Die Fachbereiche sollten sich gemeinsam darauf verständigen, das Labor jeweils zu paritätischen Anteilen zu tragen.

Selbstverständlich muss eine solche Einrichtung jedoch nicht alleine aus der Hochschule getragen werden - üblicherweise gibt es gute Möglichkeiten, die eingesetzten Mittel durch das Einwerben Drittmitteln im Rahmen einer Mischfinanzierung zu stützen.
Sobald der Bereich innerhalb der Beuth Hochschule errichtet ist und nach außen sichtbar wird, können weitere Finanzierungsquellen über private Drittmittel (z.B. aus Industrie-Kooperationen) oder öffentliche Drittmittel (Förderungen) eingeworben werden. Üblicherweise bietet ein FabLab gute Anknüpfungspunkte für Akteure und Netzwerke außerhalb der Hochschule.

In jedem Fall ist es auch zwingend notwendig und sicher eine ständige Herausforderung, die Unabhängigkeit des hochschuleigenen FabLab von einzelnen externen Fördermittel-Gebern zu bewahren. Das sollte jedoch realisierbar sein, so lange nicht einzelnen Kooperationspartnern ein übermäßiges Gewicht zugestanden wird und auf eine sinnvolle Verteilung der Kooperationen geachtet wird.


\subsection{Zusätzliche Ausstattung durch den Verein / Sponsoren}
Im Rahmen seiner Möglichkeiten muss sich der unterstützende Verein selbstverständlich an der Ausstattung des Labors beteiligen. Das muss eine primäre Zielsetzung des Vereins sein.
Im Rahmen der Registrierung des Vereins an der Beuth Hochschule muss festgelegt werden, ob die durch den Verein beschafften Gegenstände unter bestimmten Nutzungsvereinbarungen an die Beuth Hochschule übergehen - oder mit einer umgekehrten Nutzungsvereinbarung im Verein verbleiben. Die zu gestaltenden Nutzungsvereinbarungen müssen klar, einhaltbar, unkompliziert und verständlich sein.
Genaue Regelungen dazu müssen entwickelt und festgelegt werden und können im Rahmen der Registrierung gut dokumentiert werden. Im Rahmen dieser Regelungen muss auf eine angemessene Berücksichtigung der außerhalb von Lehrveranstaltungen für die Studierenden geleisteten Aufwände geachtet werden.


\subsection{Buchhaltung}
Generell wichtig ist eine saubere Buchhaltung, die klar und transparent einen Überblick über die eingesetzten Ressourcen und deren Quellen (Hochschule / Verein / Transfer-GmbH / ...), Senken (Maschinen / Verbrauchsmittel / Personal) und Nutzung (Studierende / Forschung / Kooperationspartner / Zivilgesellschaft) darstellt.

\subsection{taktische Reserve für Schäden}
In dem angedachten Betriebsmodus werden Maschinen und Ausrüstungsgegenstände sicher schneller verschleißen als bisher. Das muss allen Beteiligten klar sein und dem sollte von vorneherein Rechnung getragen werden.
Zum einen muss (vor allen Dingen finanziell) sichergestellt sein, dass verschlissene Ausrüstung und defekte Maschinen sicher und zeitnah ersetzt werden. Zum anderen sollte niemand überrascht sein wenn tatsächlich etwas kaputt geht.

Selbstverständlich muss im Betrieb durch die Nutzer darauf geachtet und besonders in den Einweisungen darauf hingewiesen werden, dass mögliche Schäden so gering wie möglich zu halten sind und dass auf einen schonenden Betrieb und eine möglichst lange Lebensdauer der Maschinen hinzuwirken ist.


\subsubsection{Maschinen-Versicherung???}
Evtl. kann der Unterstützer-Verein -- falls seine Leistungsfähigkeit das ermöglicht -- in Einzelvereinbarungen dafür eintreten, einzelne Maschinen gegen durch externe Nutzer verursachte Schäden zu versichern.

% \subsection{StartUpLab@FH}
%Mit einer Förderung durch das BMBF innerhalb des StartUpLab@FH-Programms ist das Konzept in einer ganz neuen Qualität umsetzbar. Der entsprechende Antrag ist abgegeben. Sollte die Förderung zur Verfügung stehen, kann das Konzept in seiner vollen Tragkraft umgesetzt werden und -- befreit von den aktuellen Realitäten der Berliner Haushaltspolitik und gegenläufigen Hochschul-Strukturen -- aufblühen, um einen echten Mehrwert für die Studierenden aller Fachbereiche und insbesondere die Nutzer des GründerScouts zu leisten.

\section{Sicherheit}
Generell ist nicht nur rechtlich geboten, sondern auch intrinsisch wünschenswert, dass bei allem Experimentieren, Lernen und evtl. auch ''an die Grenzen bringen`` von Technologien niemand zu schaden kommt. Daher muss Sicherheit im Labor generell immer und durch jeden mitgedacht und berücksichtigt werden. 

\samepage Um das zu erreichen, müssen nicht nur die Labormitarbeiter, sondern auch jeder einzelne Nutzer stets auf sich und andere achten. Das beinhaltet auch der Grundsatz:
\begin{quote}\emph{''Be excellent to each other.``}\end{quote}


Bei allen notwendigen Sicherheitsmaßnahmen soll dennoch gewährleistet bleiben, dass der Laborbereich zunächst ein Bereich ist, in dem die verschiedensten Nutzer und Gäste sich willkommen fühlen und nicht von vorne herein durch eine Flut von Sicherheitsbelehrungen erschlagen werden.

\subsection{Zutrittskontrolle}
Eine wirkungsvolle Zutrittskontrolle muss - unabhängig vom Betrieb als FabLab - wie bereits lange beantragt realisiert werden. In dem Bereich befinden sich Maschinen und Chemikalien, die ohne Einweisung ein erhebliches und derzeit nicht sinnvoll kontrollierbares Risiko für die Gesundheit und körperliche Unversehrtheit aller anwesenden Personen darstellen. Der Laborbereich ist aktuell vollständig ungesichert und wird regelmäßig von Personen betreten, die im Bereich nichts zu suchen haben. Um dieses Risiko auszuschalten und sinnvoll zu gewährleisten, dass die Räume nur mit entsprechender Einweisung genutzt und betreten werden können, ist eine wirkungsvolle Schließanlage am Eingang zum Labortrakt dringend notwendig. Zusätzlich kann die Maßnahme helfen, das doch aktuell leider erhebliche Diebstahlsrisiko für die Laborausstattung und das Eigentum der Labornutzer zu verringern. Die Maßnahme ist mit sämtlichen im Flachbau aktiven Akteuren\footnote{Namentlich: mit Prof. Buchgeister} abgesprochen und wird auch von SI/UmI befürwortet.

\subsection{Maschinensicherheit}
Die Arbeitssicherheit der Maschinen muss auf dem aktuellen Stand der Technik hergestellt werden.
Ein Betrieb der aktuell im Labor vorhandenen Drehbank ist mit der beabsichtigten Öffnung des Labors nicht vereinbar. Die Drehbank ist ein essentieller Teil der Laborausstattung und kann nicht einfach ''gestrichen`` werden. Daher muss im Zuge der Labor-Öffnung eine arbeitssichere Drehbank beschafft werden. Gleiches gilt für die Standbohrmaschine und eine bereits verschrottete Bandsäge.

\subsection{Einweisung in die Laborarbeit}
Bevor der Laborbereich genutzt wird, muss in jedem Fall eine strukturierte Einweisung in die Labor-Regeln erfolgen und dokumentiert werden. Ohne eine solche Einweisung darf keine Schließkarte für die  Zugangstür zum Flachbau ausgehändigt werden.

\subsection{Einweisung in einzelne Maschinen}
Die Arbeit mit bestimmten Maschinen und Chemikalien (vor allen Dingen die Metallbearbeitungsmaschinen) darf nicht ohne separate Einweisung erfolgen. Diese Einweisung ist durch den betreuenden Professor oder den Laboringenieur (in gemeinsam festgelegten Fällen evtl. auch durch Dritte) erfolgen und dokumentiert werden. 

Jede Einweisung ist nach maximal einem Jahr ebenfalls dokumentiert zu wiederholen bzw. aufzufrischen.

\subsection{Maschinenzugangskontrolle}
Um eine Nutzung ohne Einweisung wirkungsvoll zu unterbinden, müssen entsprechende Maschinen stromlos geschaltet werden. Die Maschinen dürfen nur durch Personen mit entsprechender Qualifikation und dokumentiert gültiger Einweisung freischaltbar sein.
Der Bediener ist für die Maschine verantwortlich, bis sie in sauberem Zustand wieder freigegeben wird. Auch andere Arbeitsplätze sollen zur Sicherstellung von Ordnung und Sauberkeit im Labor mit einem System ausgestattet werden, das eine Zuordnung der aktuell für den Arbeitsplatz verantwortlichen Person sicherstellt, bis der Arbeitsplatz wieder sauber und aufgeräumt freigegeben wird.

Das entsprechende System wird derzeit von Studierenden unter Betreuung durch den Laboringenieur in Kooperation mit dem FabLab der Uni Siegen entwickelt und muss vor einer verstärkten Nutzung des Laborbereiches in Betrieb genommen, getestet und stabil sein.
Die Material-Kosten belaufen sich je nach Komplexität auf ca. 15 \euro\ bis 60 \euro\ je Maschine / Arbeitsplatz.

Da dieses System ohne eine Verarbeitung von personenbezogenen Daten nicht realisierbar ist, muss eine datenschutzkonforme Ausgestaltung unter Abwägung der Interessen der Hochschule und der Nutzer beachtet werden.


\section{Offene Punkte}
Neben der angesprochenen Zugangskontrolle und Maschinensicherheit sind jedoch weitere Dinge zu klären, bevor der Laborbereich als FabLab geöffnet werden kann. 

\subsection{Raumsituation}
Freie Räume sind in der Beuth Hochschule für Technik bekannterweise Mangelware. Dennoch sind für den Aufbau eines FabLab natürlich dedizierte Flächen zwingend notwendig.

Unterschiedliche Aktivitäten innerhalb eines FabLab bedeuten zudem deutlich unterschiedliche Anforderungen an die genutzten Räume - sich konzentriert zu unterhalten ist in einer Werkstatt neben der laufenden Kreissäge nahezu unmöglich. Es müssen daher Voraussetzungen geschaffen werden die ermöglichen, dass sowohl praktische Tätigkeit (Kreissägen) als auch von Konzentration geprägte Tätigkeiten (Brainstorming / Gespräche / Veranstaltungen) gegenseitig ungestört Raum finden. 

Zum aktuellen Zeitpunkt stellt sich folgende Aufteilung als sinnvoll dar:
\subsubsection{Veranstaltungs- und Brainstorming-Bereich}
Es wird noch ein Raum für diesen Bereich benötigt. Der Raum sollte multifunktional und flexibel einrichtbar sein, eine gute Innen- und Außenwirkung gewährleisten und für folgende Arten von Veranstaltungen geeignet sein:
\begin{itemize}
 \item Workshops und Brainstorming-Veranstaltungen.
 \item Gruppenveranstaltungen.
 \item Veranstaltungen mit Aufzeichnung und / oder Online-Streaming.
 \subitem Podiumsveranstaltungen mit einem oder mehreren Rednern.
 \subitem Diskussionsveranstaltungen, auch mit interkativer Einbindung eines Publikums.
 \subitem Präsentationsveranstaltungen (z.B. Projekte/Produkte/Forschungsergebnisse/...) mit teilweise hochrangigen externen Gästen.
\end{itemize}
Evtl. kommt für die Anfangszeit die Beuth-Halle in Frage, falls die Nutzung dort nicht auf Hindernisse stößt. Auch der Raum E00 im Eingangsbereich des Haus Bauwesen war im Gespräch, hier konnte jedoch kein Konsens mit der Studierendenschaft erzielt werden. Falls dieser Raum tatsächlich für die Zwecke des InnoVision-Lab\footnote{''InnoVision-Lab`` ist der aktuelle Arbeitstitel des mit dem Medienbereich der BeuthBox kombinierten FabLab.} genutzt werden soll, müssen den Studierenden passende Flächen als Ersatz für den dann belegten Lernraum zur Verfügung gestellt werden. Gleichwohl soll die Fläche selbstverständlich auch bei einer Nutzung als Veranstaltungs- und Gruppenarbeitsbereich des FabLab den Studierenden nach wie vor niederschwellig zur Verfügung stehen.


\subsubsection{Werk- und Arbeitsbereich}
Im Werk- und Arbeisbereich sollen Tätigkeiten stattfinden, die ''stauben, stinken, potentiell nicht ungefährlich sind und / oder Lärm erzeugen``. Also all jene praktischen Tätigkeiten, die oft zur Erstellung eines Prototypen dringend notwendig sind - jedoch ein ungestörtes und konzentriertes Arbeiten im direkten Umfeld unmöglich machen.


Primär ist für diesen Bereich das aktuelle FVM-Labor im FB VII zu nennen. Es handelt sich um ein als Werkhalle aufgebautes Gebäude mit stabilen und belastbaren Böden\footnote{Einige Maschinen in einem FabLab haben ein Gewicht von teilweise 2,5 Tonnen - die Böden in diesem Teil des FabLab müssen für diese Last geeignet sein.}. Viele der im FabLab geplanten Einrichtungen und Aktivitäten sind dort bereits vorhanden. Durch den Betrieb als FabLab würde die Auslastung dieser Flächen (bisher fast ausschließlich durch wenige Studiengänge im FB VII) deutlich steigen und die Fläche. Gleichzeitig verliert der aktuell alleinige Nutzer - der FB VII - selbstverständlich nicht die Möglichkeit, die Einrichtungen zu nutzen. Der FB VII gewinnt durch die gemeinsame Nutzung mit anderen Fachbereichen als FabLab eher noch dazu, da in dem Bereich im gemeinsamen Betrieb Ressourcen konzentriert werden und Synergien zum Tragen kommen. Aufgrund der fachlichen Ausrichtung wird auch im Betrieb als FabLab der FB VII einer der Hauptnutzer des dann neu geschaffenen Bereiches sein.

\subsection{Werkzeugausstattung}
Derzeit ist für mehrere parallel arbeitende Arbeitsgruppen deutlich zu wenig geeignetes Werkzeug vorhanden. Unabhängig von der Öffnung als FabLab muss daher für die Nutzer geeignetes Werkzeug in entsprechender Menge beschafft werden.

\subsection{Maschinenausstattung}
Ein Teil der vorhandenen Maschinen ist derzeit nicht arbeitssicher und wurde stillgelegt. Diese Maschinen sind jedoch essentiell wichtig sowohl für die im Labor derzeit (und auch zukünftig) stattfindende Forschung, als auch für eine umfassende, praktische Lehre und gerade auch für den Betrieb als FabLab. Daher müssen diese Maschinen in einer arbeitssicheren Variante nachbeschafft werden und zur Verfügung stehen.

\end{document}
